%%%%%%%%%%%%%%%%%%%%%%%%%%%%%%%%%%%%%%%%%%%%%%%%%%%%%%%%%%%%%%%%%%%%%%%%%%%%%%%%%%%%%%%%%%%%%%%%%%%%%%%%%%%%%
% % % % % % % % % % % % % % % % % % % % % % % % % % % % % % % % % % % % % % % % % % % % % % % % % % % % % % % 
% = = = = = = = = = = = = = = = = = = = = = = = = = = = = = = = = = = = = = = = = = = = = = = = = = = = = = =
%
% This is where the packages are declared

\documentclass[]{article}

\usepackage[english]{babel}
\usepackage[utf8]{inputenc}
\usepackage{amsmath}
\usepackage{graphicx}
\usepackage[colorinlistoftodos]{todonotes}

\begin{document}

\title{Man, Economy and State}
\author{Book Club Notes}
\date{\today}

\maketitle

\begin{abstract}
It will help me.
\end{abstract}

%------------------------------------------------------------------------------------------------------------
%------------------------------------------------------------------------------------------------------------
\section{Fundamentals of Human Action}
%------------------------------------------------------------------------------------------------------------
%------------------------------------------------------------------------------------------------------------

\begin{itemize}
    \item Concept/axiom of human action: \\ purposeful behaviour, intentions are core of it. Praxeology not saying WHY or HOW it comes from
    \item Understanding human action of others requires projection of the concept of action of others onto self
    \item Groups don't act; individuals do, but man is still a social being
    \item HA takes place in time (cornerstone of AE). Things happen sequentially, not simultaneously
    \item Time is the main scarce resource
    \item Acting is done in accordance with the end
    \item Preferences are revealed through action
    \item All means are scarce
    \item Uncertainty of the future: idea of the end may not fully realize thanks to errors
    \item Consumer vs producer goods; land and labour are original factors
    \item Consumer goods are valued through utility satisfied; producer goods are valued through utility satisfied by consumer goods they produce
    \item Psychic revenue
    \item Humans are forward-looking, using past as guidance
    \item No interpersonal measurement of utility is possible
    \item Universal law of time preference
    \item All action is an attempt to exchange a less satisfactory state of affairs for a more satisfactory one
    \item Humans value units of a good, not whole batches (stocks); units are subjective and not necessarily physical. Interchangibility is key
    \item One can't divide some concept of 'value' of stock over number of units; marginal utility
    \item Law of returns: with the quantity of complementary factors held constant, there always exists some optimum amount of the varying factor (average unit product)
    \item Factors of production have different degrees of specificity (cigar making machine vs. tobacco)
    \item More specific factors of production tend to drop harder in price when value of product drops (empirical, not praxeological)
    \item People value leisure; people who work 10\% more earn 40\% more money, because the disutility of labour (= utility of leisure) rises with every extra hour worked (disutility of labor = empirical observation)
    \item Satisfaction from the job exists and is also valued as a consumer good
    \item Capital goods can be produced only by act of saving and investing (i.e. lowering time preference); to produce a fish-catching net one needs to save some fish first. When building, one will consume capital
    \item Capital formation depends on time preference; lower time preference allows for more capital goods
    \item Any individual at any time can accumulate capital, leave it intact or consume it
\end{itemize}

\newpage

\section{Direct Exchange}

\begin{itemize}
    \item Introduction of another individual
    \item Different types of interaction between individuals, including slavery
    \item Voluntary slavery is an oxymoron
    \item Any forced exchange benefits only one person
    \item Society: any continuing pattern of interpersonal exchange
    \item Types of society: hegemonic, contractual
    \item Two types of exchanges: autistic and interpersonal
    \item Voluntary exchange will happen only if both parties believe they'll benefit from it, i.e. both parties value the goods of the other party more than they value goods they offer
    \item Law of marginal utility
    \item Exchange will eventually stop, because of Law of marginal utility
    \item Use value: marginal utility attained from consumption
    \item Exchange value: 
    \item Exchange will take place when exchange value $>$ use value
    \item (ways of appropriating property etc.)
    \item Division of labour is made possible by differences in human beings
    \item Law of association
    \item Law of comparative advantange: absolute vs. relative advantange
    \item Division of labour can only occur with indirect exchange
    \item Direct exchange problem: double coincidence of wants
    \item Psychic revenue stressed again; ultimately it's not physical properties that's important, but psychic benefit. There are some thing one can't quantify
    \item Minimum selling price of seller > maximum buying price of buyer in order to exchange to take place
    \item With 2 people, we may not be able to determine exact price, only a range and price will be also determined by bargaining skill
    \item Equillibrium price is one where quantity demanded is quantity sold
    \item Formation of equillibrium price on a market
    \item Austrian term: "final state of rest"
    \item One price tends to be established on a market, near the intersection of supply and demand curves
    \item Demand curve must be vertical or rightward-sloping as price increases
    \item Supply curve must by 
    \item It's essentially people on the margin that determine equillibrium price
    \item Psychic gain can't be measured
    \item Elasticity of demand: 
    \item The less substitutes a good has, more likely for demand to be inelastic, etc.
    \item Elasticity of supply is not a meaningful concept
    \item Speculation: anticipation of change in price of a commodity
    \item Alternate framework to understand formation of equillibrium price: stock and total demand to hold
    \item Equillibrium price equates these two
    \item The concept obscures the volume of exchanges that take place
    \item Differences between moving along demand/supply curves vs. change in demand/supply curves and effects of that
    \item Markets are constantly changing
    \item Changes in demand/supply curves change relative exchange value of goods
    \item Formation of market supply and schedules
    \item Producers are constantly adjusting output to meet demand
    \item Demand schedules, ergo future prices, can only be anticipated
    \item The bearing of uncertaintny is the key concept of entrepreneurship in AE
    \item Anticipating whole demand schedule is ~
    \item Types of exchanges (pg. 169)
    \item Appropriation of land: diss on Georgism as man needs to own land to produce things etc
    
\end{itemize}

\newpage

\section{Indirect Exchange}

\begin{itemize}
    \item Neoclassical models: money as a commodity doesn't have value in itself
    \item Direct exchange is limited by double coincidence of wants
\end{itemize}

\end{document}
