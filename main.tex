%%%%%%%%%%%%%%%%%%%%%%%%%%%%%%%%%%%%%%%%%%%%%%%%%%%%%%%%%%%%%%%%%%%%%%%%%%%%%%%%%%%%%%%%%%%%%%%%%%%%%%%%%%%%%
% % % % % % % % % % % % % % % % % % % % % % % % % % % % % % % % % % % % % % % % % % % % % % % % % % % % % % % 
% = = = = = = = = = = = = = = = = = = = = = = = = = = = = = = = = = = = = = = = = = = = = = = = = = = = = = =
%
% This is where the packages are declared

\documentclass[]{article}

\usepackage[english]{babel}
\usepackage[utf8]{inputenc}
\usepackage{amsmath}
\usepackage{graphicx}
\usepackage[colorinlistoftodos]{todonotes}

\begin{document}

\title{Man, Economy and State}
\author{Book Club Notes}
\date{\today}

\maketitle

\begin{abstract}
It will help me.
\end{abstract}

%------------------------------------------------------------------------------------------------------------
%------------------------------------------------------------------------------------------------------------
\section{Fundamentals of Human Action}
%------------------------------------------------------------------------------------------------------------
%------------------------------------------------------------------------------------------------------------

\begin{itemize}
    \item Concept/axiom of human action: \\ purposeful behaviour, intentions are core of it. Praxeology not saying WHY or HOW it comes from
    \item Understanding human action of others requires projection of the concept of action of others onto self
    \item Groups don't act; individuals do, but man is still a social being
    \item HA takes place in time (cornerstone of AE). Things happen sequentially, not simultaneously
    \item Time is the main scarce resource
    \item Acting is done in accordance with the end
    \item Preferences are revealed through action
    \item All means are scarce
    \item Uncertainty of the future: idea of the end may not fully realize thanks to errors
    \item Consumer vs producer goods; land and labour are original factors
    \item Consumer goods are valued through utility satisfied; producer goods are valued through utility satisfied by consumer goods they produce
    \item Psychic revenue
    \item Humans are forward-looking, using past as guidance
    \item No interpersonal measurement of utility is possible
    \item Universal law of time preference TODOTODO
    \item All action is an attempt to exchange a less satisfactory state of affairs for a more satisfactory one
    \item Humans value units of a good, not whole batches (stocks); units are subjective and not necessarily physical. Interchangibility is key
    \item One can't divide some concept of 'value' of stock over number of units; marginal utility
    \item Law of returns: with the quantity of complementary factors held constant, there always exists some optimum amount of the varying factor (average unit product)
    \item Factors of production have different degrees of specificity (cigar making machine vs. tobacco)
    \item More specific factors of production tend to drop harder in price when value of product drops (empirical, not praxeological)
    \item People value leisure; people who work 10\% more earn 40\% more money, because the disutility of labour (= utility of leisure) rises with every extra hour worked (disutility of labor = empirical observation)
    \item Satisfaction from the job exists and is also valued as a consumer good
    \item Capital goods can be produced only by act of saving and investing (i.e. lowering time preference); to produce a fish-catching net one needs to save some fish first. When building, one will consume capital
    \item Capital formation depends on time preference; lower time preference allows for more capital goods
    \item Any individual at any time can accumulate capital, leave it intact or consume it
\end{itemize}

\end{document}